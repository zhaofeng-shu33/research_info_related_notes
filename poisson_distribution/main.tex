\documentclass{article}
\usepackage{amsmath}
\usepackage{graphicx}
\DeclareMathOperator{\Pois}{Pois}
\title{Poisson Distribution}
\author{zhaof17 }
\date{January 2021}

\begin{document}

\maketitle

\section{Introduction}
An important property for Poisson distribution.
Suppose $P_1 \sim \Pois(c)$, $P_2 \sim \Pois(d)$.
And $P_{\lambda}$ is defined as 
$$
P_{\lambda}(x) = \frac{P_1^{1-\lambda}(x) P_2^{\lambda} (x)}
{\sum_{x \in \mathcal{X}}P_1^{1-\lambda}(x) P_2^{\lambda} (x)}
$$
\begin{enumerate}
    \item $P_{\lambda} \sim \Pois(c^{1-\lambda} d^{\lambda})$
    \item $D_{\mathrm{KL}}(P_{\lambda}||P_1) = c^{1-\lambda}d^{\lambda}\log(\frac{d}{c})^{\lambda} + c-c^{1-\lambda}d^{\lambda}$
\end{enumerate}
The second property is proved using the KL divergence
of two Poisson distributions. See \cite{kl}
\section{Side Information of Abbe}
This section discusses theoretical deduction and experimental results for \cite{abbe}.

Let $c_1 = \frac{a}{2}, c_2=\frac{b}{2}$. $p_0, p_1$ are two
discrete distributions.
The Abbe's conclusion for side information is solving the following
optimization problem:
\begin{equation}\label{eq:lambda}
    \min_{\lambda \in [0,1]} c_1^{1-\lambda}c_2^{\lambda} +
    c_2^{1-\lambda}c_1^{\lambda} + \gamma \log(\sum_{x\in \mathcal{X}}p^{1-\lambda}_0(x) p^{\lambda}_1(x))
\end{equation}
Let $p_{\lambda}$ be defined as
$$
p_{\lambda}(x) = \frac{p_0^{1-\lambda}(x) p_1^{\lambda} (x)}
{\sum_{x \in \mathcal{X}}p_0^{1-\lambda}(x) p_1^{\lambda} (x)}
$$
After solving $\lambda$ from \eqref{eq:lambda}. The theoretical
threshold is given by
\begin{align}
    I_+ =\lambda (c_1^{1-\lambda}c_2^{\lambda} -
    c_2^{1-\lambda}c_1^{\lambda})\log\frac{c_2}{c_1}+c_1+c_2\notag \\
    -c_1^{1-\lambda}c_2^{\lambda} -
    c_2^{1-\lambda}c_1^{\lambda}+\gamma D_{\mathrm{KL}}(p_{\lambda}||p_0)>1
\end{align}
When $\lambda=\frac{1}{2}$ makes $D_{\mathrm{KL}}(p_{\lambda}||p_0)=
D_{\mathrm{KL}}(p_{\lambda}||p_1)$, the threshold of Abbe is the same as
us.

\subsection{Experiment Result}
\begin{figure}
    \centering
    \includegraphics[width=0.6\textwidth]{fig.png}
    \label{fig:my_label}
\end{figure}

The red line is our theoretical line
given by $a=(\sqrt{2 - \gamma D_{1/2}(p_0||p_1)} + \sqrt{b})^2$,
which is strictly over the black region. It tells that
exact recovery condition of Abbe is more stronger than us.
\subsection{Theoretical Equivalence}
Recall the definition of $g(\epsilon)$ as
	\begin{equation}\label{eq:gab}
	g(a,b,\epsilon) = a + b - \sqrt{\epsilon^2 + 4ab} + \epsilon \log \frac{\epsilon + \sqrt{\epsilon^2 + 4ab}}{2b}
	\end{equation}
	
We show that $I_+$ is equal to $\theta_1^*$.
\begin{align}
\theta^*_1 &= \min_{\widetilde{X}_1} \gamma D(p_{\widetilde{X}_1}|| p_0)+ \frac{1}{2} g(a,b, 2\epsilon)  \label{eq:theta}\\
\epsilon &= \gamma \frac{D(p_{\widetilde{X}_1} || P_1) - D(p_{\widetilde{X}_1} || P_0) }{\log a /b}\label{eq:equal}
\end{align}
We use Lagrange multiplier to solve \eqref{eq:theta}.
Let
$$
L(p_{\widetilde{X}_1},\epsilon, \lambda)
=\gamma D(p_{\widetilde{X}_1}|| p_0)+ \frac{1}{2} g(a,b, 2\epsilon) - \lambda(\epsilon \log\frac{a}{b}-\gamma
D(p_{\widetilde{X}_1} || P_1) + \gamma D(p_{\widetilde{X}_1} || P_0))
$$
It is equivalent to minimize
$(1-\lambda)D(p_{\widetilde{X}_1} || P_0) +
\lambda D(p_{\widetilde{X}_1} || P_1) $, from
which we get
\begin{equation}\label{eq:p12}
p_{\widetilde{X}_1}(x) = \frac{p_0^{1-\lambda}(x)p_1^{\lambda}(x)}{\sum_{x \in \mathcal{X}}p_0^{1-\lambda}(x) p_1^{\lambda} (x)}.
\end{equation}
From $\frac{\partial L(p_{\widetilde{X}_1},\epsilon, \lambda)}{\partial \epsilon}=0$ and taking \eqref{eq:p12}
into \eqref{eq:equal}, we get
\begin{align*}
    \lambda \log \frac{a}{b}
    & = \log \frac{\epsilon + \sqrt{\epsilon^2+ab}}{b} \\
    \epsilon \log \frac{a}{b}
    & = \gamma\frac{\sum_{x \in \mathcal{X}}p_0^{1-\lambda}(x) p_1^{\lambda} (x)\log \frac{p_0(x)}{p_1(x)}}{\sum_{x \in \mathcal{X}}p_0^{1-\lambda}(x) p_1^{\lambda} (x)}
\end{align*}
After cancelling $\epsilon$
from the above two equations, we can get a single equation
for $\lambda$:
\begin{equation}
    \frac{1}{2}\log\frac{a}{b}
    (a^{\lambda} b^{1-\lambda}
    -a^{1-\lambda} b^{\lambda})
    + \gamma \frac{\sum_{x \in \mathcal{X}}p_0^{1-\lambda}(x) p_1^{\lambda} (x)\log \frac{p_1(x)}{p_0(x)}}{\sum_{x \in \mathcal{X}}p_0^{1-\lambda}(x) p_1^{\lambda} (x)}=0
\end{equation}
which is the derivative of \eqref{eq:lambda}.
Then by simple computation we have
$I_+ = \theta_1^*$.

\begin{thebibliography}{9}
\bibitem{kl} https://stats.stackexchange.com/questions/145789/kl-divergence-between-two-univariate-poisson-distributions
\bibitem{abbe} Asadi, Amir R., Emmanuel Abbe, and Sergio Verdú. "Compressing data on graphs with clusters." 2017 IEEE International Symposium on Information Theory (ISIT). IEEE, 2017.
\end{thebibliography}
\end{document}
