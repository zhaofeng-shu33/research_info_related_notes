\documentclass{ctexbeamer}
\usepackage{graphicx}
\usepackage{xcolor}
\usepackage{amssymb}
\usepackage{amsfonts}
\usepackage{pifont}
\usepackage{bm}
\usepackage{subcaption}
\def\E{\mathbb{E}}
\def\R{\mathbb{R}}
\DeclareMathOperator{\Vol}{Vol}
\newif\ifbeamer
\beamertrue

%further improvement
%background intro added Chen Chuan's work
%simulation add hierarchical method
%find suitable dataset to apply info-clustering
%Yang Li's suggestion: compared with other clustering method, the threshold is the same or not
%early stopping technique, complexity from n -> log(n)
\title{Elements of Ensemble Theory}
%\author{Feng Zhao}
\date{\today}
\begin{document}
\begin{frame}
	\titlepage
\end{frame}
\begin{frame}{Outline}
    \tableofcontents
\end{frame}
\section{2.1 Phase space of a classical system}

\begin{frame}
\frametitle{经典系统的相空间}
\begin{itemize}
    \item 系综的概念:同一观测时间不同系统的集合
    \item Hamiltonian 系统 $H(q_i,p_i)$
    \item Hamiltonian 方程描述了 Hamiltonian 系统各粒子的位置和动量如何演化
    \item Hamiltonian 量相当于系统的能量
    \item 物理量 $f(q, p)$ 的系综平均:$\langle f \rangle$
    \item 稳态的系综:密度不显示依赖于时间 $\phi(q, p)$
\end{itemize}

\end{frame}
\section{2.2 Liouville's theorem and its consequences}
\begin{frame}{刘维尔定理及其结果}
    \begin{itemize}
        \item 刘维尔定理:相空间的密度随着相空间的点的运动轨迹不发生变化
        \item 结果:稳态系综的泊松括号为零
        \item 微正则系综:相空间的密度函数是常数
        \item 正则系综
    \end{itemize}
\end{frame}

\section{2.3-2.4 The microcanonical ensemble}
\begin{frame}{微正则系综}
    \begin{itemize}
        \item 某物理量的系综平均等于给定系统对时间求平均
        \item 相空间的体积$\omega$
        与微观系统状态数$\Gamma$
        成正比,比例系数为 $\omega_0$
        \item 通过$N$个粒子的理想气体系统和单粒子的谐振子系统说明 $\omega_0=h^{\mathcal{N}}$
    \end{itemize}
    \begin{block}{符号说明}
    \begin{itemize}
        \item $h$: Planck constant
        \item $\mathcal{N}$: 系统自由度,一个自由度对应一个$(q_i, p_i)$
    \end{itemize}
    \end{block}
\end{frame}
\section{2.5 Quantum states and the phase space}
\begin{frame}{量子态和相空间}
    转换系数$\omega_0$ 与测不准原理有关
\end{frame}
\end{document}
