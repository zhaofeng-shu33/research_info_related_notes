\documentclass{beamer}
\usepackage{graphicx}
\usepackage{xcolor}
\usepackage{amssymb}
\usepackage{amsfonts}
\usepackage{pifont}
\usepackage{subcaption}
\def\E{\mathbb{E}}
\def\R{\mathbb{R}}
\DeclareMathOperator{\Vol}{Vol}
\newif\ifbeamer
\beamertrue

%further improvement
%background intro added Chen Chuan's work
%simulation add hierarchical method
%find suitable dataset to apply info-clustering
%Yang Li's suggestion: compared with other clustering method, the threshold is the same or not
%early stopping technique, complexity from n -> log(n)
\title{The Statistical Basis of Thermodynamics}
%\author{Feng Zhao}
\date{\today}
\begin{document}
\begin{frame}
	\titlepage
\end{frame}
\begin{frame}{Outline}
    \tableofcontents
\end{frame}
\section{1.1 The macroscopic and the microscopic states}

\begin{frame}
\frametitle{The macroscopic and the microscopic states}
\begin{block}{The number of microstates: $\Omega(N,V,E)$}
\begin{itemize}
    \item $N$ is the number of particles
    \item $V$ is the volume of system
    \item $E$ is the system energy
\end{itemize}
\end{block}
\begin{block}{Investigate how $\Omega$ is related with other thermodynamic quantities}
    such as
\begin{itemize}
    \item the temperature $T$
    \item the pressure $P$
    \item the chemical potential $\mu$
\end{itemize}
\end{block}

\end{frame}
\section{1.2-1.3 Physical significance of the number $\Omega(N,V,E)$}
\begin{frame}{Physical significance of the number $\Omega(N,V,E)$}
    The partial derivatives of $\Omega(N,V,E)$ gives the expression of $\mu, P, T$
    respectively.
    \begin{block}
    {Concepts from thermodynamics}
    \begin{itemize}
        \item entropy $S=k\ln \Omega$
    \item enthalpy $H$
    \item Gibbs free energy $G$, Helmholtz free energy $A$,
    \item heat capacity $C_V$ (constant volume) and $C_P$ (constant pressure)
    \end{itemize}
    \end{block}
\end{frame}

\section{1.4 The classical ideal gas}
\begin{frame}{The classical ideal gas}
    \begin{itemize}
        \item 
    Give derivation of $\ln \Omega$ and derive known formulas by taking derivatives.

    \item  Hard: Computing the exact number of particles whose total energy is $E$.
    
    \item  Easy: Compute the number of particles whose total energy is less than $E$
    
    \item The two quantities are approximately equal when $N$ is very large.    
    \end{itemize}
\end{frame}
\section{1.5 The entropy of mixing and the Gibbs paradox}
\begin{frame}{The entropy of mixing and the Gibbs paradox}
    \begin{itemize}
        \item The extensive property of $S$.

        \item Mixing the same gas is different with mixing two different ideal gases.
    
        \item The former is reversible, $\Delta S = 0$.
    
        \item The latter is irreversible, $\Delta S > 0$.
    
    \end{itemize}
\end{frame}
\section{1.6 The "correct" enumeration of the microstates}
\begin{frame}{The "correct" enumeration of the microstates}
    \begin{itemize}
        \item Particles with different energies are indistinguishable. 
    
    \item Correcting $\Omega$ computed previously by dividing it by $N!$
    \end{itemize}
\end{frame}
\end{document}
