\documentclass{article}
\usepackage{ctex}
\usepackage{bm}
\usepackage{bbm}
\usepackage{amsmath}
\usepackage{amsthm}
\usepackage{algorithm,algorithmic}
\usepackage{url}
\usepackage{amssymb}
\usepackage{mathtools}
\usepackage{hyperref}
\usepackage{xcolor}
\newtheorem{theorem}{Theorem}
\newtheorem{example}{Example}
\newtheorem{lemma}{Lemma}
\def\E{\mathbb{E}}
\def\R{\mathbb{R}}
\DeclareMathOperator{\erf}{erf}
\title{Unrelated materials}
\begin{document}
\maketitle
\section{A high dimensional ball}
Suppose $X$ is an n dimensional random gaussian vector
sampled from
$\mathcal{N}(\bm{0}, I_n)$. The upper bound of probability that it
falls within a hyperball with radius $R$ is computed
as follows:
\begin{align*}
    P(||X||\leq R) &= \int_{x_1^2 + \dots + x_n^2 \leq R^2}
    \frac{1}{(2\pi)^{n/2}}
    e^{-\frac{x_1^2 + \dots x_n^2}{2}}
    dx_1 \dots dx_n\\
    &= \frac{S_{n-1}(1) }{(2\pi)^{n/2}}
    \int_0^R r^{n-1} e^{-r^2/2}dr
    \\
    &=\frac{2\pi^{n/2}}{(2\pi)^{n/2}\Gamma(n/2)}
    \int_0^R r^{n-1} e^{-r^2/2}dr \\
    &=\frac{2\pi^{n/2}}{(2\pi)^{n/2}\Gamma(n/2)}
    \int_0^R r^{n-1} e^{-r^2/2}dr \\
    &=\frac{1}{\Gamma(n/2)}
    \int_0^{\frac{R^2}{2}} x^{\frac{n}{2} - 1} e^{-x}dx\\
    &=\frac{\gamma(n/2, \frac{R^2}{2})}{\Gamma(n/2)}
\end{align*}
We can use the recurrence relation for the
incomplete gamma function $\gamma(s,x)$:
\begin{equation}
    \gamma(s+1, x)
    = s\gamma(s, x) - x^s e^{-x}
\end{equation}
We consider the recurrence formula for the
following sequence:
$a_{n+1} = n a_n - bk^n$.
Dividing both sides by $n!$, we obtain
$\frac{a_{n+1}}{n!} =
\frac{a_n}{(n-1)!} - b\frac{k^n}{n!}$.
Therefore, $\frac{a_n}{(n-1)!} = a_1 - b\sum_{i=1}^{n-1}
\frac{k^i}{i!}$ while $a_1 = \gamma(1, k)=1-e^{-k}$
and $b=e^{-k}$.
$\frac{a_n}{(n-1)!} = e^{-k}(e-\sum_{i=0}^{n-1} \frac{k^i}{i!})
\sim e^{-k}\frac{k^n}{n!}$.
Therefore, suppose $n$ is an even number,
$P(||X||\leq R) \sim e^{-k}\frac{(R^2/2))^n}{n!}$.
The decaying rate is faster than exponential rate.

A looser bound is obtained by replacing the hyperball
with a bounding hypercube. Then
$P(||X||\leq R) \leq P(|X_1|\leq R)^n$, which is an
exponential rate.

\section{Crofton's formula}
This part mainly consults \cite{crofton}.

Crofton's formula gives a way to compute the arc length
by integration. Suppose we have a two dimensional
parametrized curve $\gamma: [0, 1] \to \mathbb{R}^2$.
Then the arc length of $\gamma$ is
\begin{equation}\label{eq:gamma}
    len(\gamma) = \frac{1}{2}\iint n(p, \theta)
    dp d\theta
\end{equation}
where $p, \theta$ is the polar coordinate representation
of a straight line. $n(p, \theta)$ is the number of times the 
straight line intersects the curve. The integration
in \eqref{eq:gamma} is over $p\in [0, +\infty)$
and $\theta \in [0, 2\pi]$.
\begin{example}
    The perimeter of the circle $x^2+y^2=r^2$.
    Using the integration formula \eqref{eq:gamma},
    we obtain $\frac{1}{2}\int_{0}^{2\pi} d\theta \int_0^r 2 dp=2\pi r$.
\end{example}
\begin{example}
    The line segment $0\leq x \leq 1, y=0$.
    The integral is $\frac{1}{2}\int_0^1 2 dp
    \int_0^{\arccos p} d\theta =
    \int_0^1 \arccos p dp=1$.
\end{example}
\bibliographystyle{plain}
\bibliography{ref.bib}
\end{document}



