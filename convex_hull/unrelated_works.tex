\documentclass{article}
\usepackage{ctex}
\usepackage{bm}
\usepackage{bbm}
\usepackage{amsmath}
\usepackage{amsthm}
\usepackage{algorithm,algorithmic}
\usepackage{url}
\usepackage{amssymb}
\usepackage{mathtools}
\usepackage{hyperref}
\usepackage{xcolor}
\newtheorem{theorem}{Theorem}
\newtheorem{example}{Example}
\newtheorem{lemma}{Lemma}
\def\E{\mathbb{E}}
\def\R{\mathbb{R}}
\DeclareMathOperator{\erf}{erf}
\title{Unrelated materials}
\begin{document}
\maketitle
\section{A high dimensional ball}
Suppose $X$ is an n dimensional random gaussian vector
sampled from
$\mathcal{N}(\bm{0}, I_n)$. The upper bound of probability that it
falls within a hyperball with radius $R$ is computed
as follows:
\begin{align*}
    P(||X||\leq R) &= \int_{x_1^2 + \dots + x_n^2 \leq R^2}
    \frac{1}{(2\pi)^{n/2}}
    e^{-\frac{x_1^2 + \dots x_n^2}{2}}
    dx_1 \dots dx_n\\
    &= \frac{S_{n-1}(1) }{(2\pi)^{n/2}}
    \int_0^R r^{n-1} e^{-r^2/2}dr
    \\
    &=\frac{2\pi^{n/2}}{(2\pi)^{n/2}\Gamma(n/2)}
    \int_0^R r^{n-1} e^{-r^2/2}dr \\
    (x=\frac{r^2}{2})
    &=\frac{1}{\Gamma(n/2)}
    \int_0^{\frac{R^2}{2}} x^{\frac{n}{2} - 1} e^{-x}dx\\
    &=\frac{\gamma(n/2, \frac{R^2}{2})}{\Gamma(n/2)}
\end{align*}
The function $\gamma(s,x)$
is called the lower
incomplete gamma function.
Or we can write
\begin{equation}
P(||X||\geq R)
= \frac{\Gamma(n/2, \frac{R^2}{2})}
{\Gamma(n/2)}
\end{equation}
The function
$\Gamma(s,x)=\int_x^{+\infty} t^{s-1} e^{-t} dt$
is called the upper incomplete
gamma function.
For $x\to \infty$,
we have the asymptotic behavior for $\Gamma(s,x)$
as:
\begin{equation}\label{eq:incomplete_upper_gamma}
    \Gamma(s,x) \sim x^{s-1} \exp(-x)
\end{equation}

%Therefore, suppose $n$ is an even number,
$P(||X||\geq R) \sim e^{-R^2/2}\frac{(R^2/2)^{n/2-1}}{\Gamma(n/2)}$.
The decaying rate exponential.

If $n=1$, we obtain $P(X\geq x) = \frac{1}{2}P(|X|\geq x)
= \frac{1}{\sqrt{2\pi }x}\exp(-x^2/2)$, which is consistent with
the asymptotic behavior of the Gaussian tail.

\begin{proof}
    We prove \eqref{eq:incomplete_upper_gamma}
    for $s=n$ (a positive integer).
For this purpose,
we can use the recurrence relation for the
incomplete gamma function $\Gamma(s,x)$:
\begin{equation}
    \Gamma(s+1, x)
    = s\Gamma(s, x) + x^s e^{-x}
\end{equation}

We consider the recurrence formula for the
following sequence. Let $a_{n}=\Gamma(n, x),
b=e^{-x}$.
Then $a_{n+1} = n a_n + bx^n$.
Dividing both sides by $n!$, we obtain
$\frac{a_{n+1}}{n!} =
\frac{a_n}{(n-1)!} + b\frac{x^n}{n!}$.
Therefore, $\frac{a_n}{(n-1)!} = a_1 + b\sum_{i=1}^{n-1}
\frac{x^i}{i!}$ while $a_1 = \Gamma(1, x)
=e^{-x}$.
$\frac{a_n}{(n-1)!} = e^{-x}
(\sum_{i=0}^{n-1} \frac{x^i}{i!})$
.
When $x\to \infty$, the dominant term
in the summation is $x^{n-1}$. Therefore,
$\Gamma(n,x) = a_n \sim x^{n-1} e^{-x}$.

Another proof is given based on integration by parts.
This method is the first example in the textbook of
asymptotic analysis in \cite{murray2012asymptotic}.
\end{proof}

A looser bound is obtained by replacing the hyperball
with a bounding hypercube. Then
$P(||X||\leq R) \leq P(|X_1|\leq R)^n$, which is also of
exponential decaying rate.

We can also investigate $P(||X||\leq R) $
when fix $R$ while letting the dimension $n$ to $\infty$.
From (8.11.5) of \cite{nist}, we have
\begin{equation}
    P(a,z) = \frac{\gamma(a,z)}{\Gamma(a)} \sim \frac{z^a e^{-z}}{\Gamma(1+a)}
\end{equation}
When $a\to \infty$, $P(a,z) \to 0$. That is, the ball with fixed radius
has near zero volume in higher dimensions.
\section{Crofton's formula}
This part mainly consults \cite{crofton}.

Crofton's formula gives a way to compute the arc length
by integration. Suppose we have a two dimensional
parametrized curve $\gamma: [0, 1] \to \mathbb{R}^2$.
Then the arc length of $\gamma$ is
\begin{equation}\label{eq:gamma}
    len(\gamma) = \frac{1}{2}\iint n(p, \theta)
    dp d\theta
\end{equation}
where $p, \theta$ is the polar coordinate representation
of a straight line. $n(p, \theta)$ is the number of times the 
straight line intersects the curve. The integration
in \eqref{eq:gamma} is over $p\in [0, +\infty)$
and $\theta \in [0, 2\pi]$.
\begin{example}
    The perimeter of the circle $x^2+y^2=r^2$.
    Using the integration formula \eqref{eq:gamma},
    we obtain $\frac{1}{2}\int_{0}^{2\pi} d\theta \int_0^r 2 dp=2\pi r$.
\end{example}
\begin{example}
    The line segment $0\leq x \leq 1, y=0$.
    The integral is $\frac{1}{2}\int_0^1 2 dp
    \int_0^{\arccos p} d\theta =
    \int_0^1 \arccos p dp=1$.
\end{example}
\section{Spherical triangles}
We treat the  right spherical triangles. with $C=\frac{\pi}{2}$.
From Napier's rules
(See \cite{Napier}),
R7+R4: $\tan A \sin b = \cos B \tan c$.
Combined with R3: $\sin b = \sin B \sin c$.
We have $\tan A = \frac{\cot B}{\cos c}$.

Suppose we have two spheres centered at the origin, with radius $x$ and $z$ respectively.
Now there is a straight line whose distance to the origin is $y$.
We have the relationship $x<y<z$.
We consider two tangent planes to the smaller sphere.
Both planes pass through the given straight line.
We are interested in the area
on the larger sphere,
intersected by two circles.
This problem is a special case
of "the area on a sphere of the intersection of two spherical caps".
See \cite{intersection}
for detail.
The area is given as $S=2z^2(\pi - 2\beta - \alpha \cdot x/z)$.
And the ratio of this area (over the entire sphere surface)
is given as
$\frac{\pi - 2\beta - \alpha \cdot x/z}{2\pi}$.
where
\begin{align}
\cos \alpha & = \frac{2y^2 - z^2 - x^2}{
    z^2 - x^2 }\\
\tan \beta &= \frac{z}{ 
    x  \tan(\alpha/2)} = \frac{z \sqrt{y^2-x^2}}{x\sqrt{z^2-y^2}}
\end{align}
We can verify numerically that the above result is the same with 
the one \cite{orthogonal}:
\begin{equation}
    A=\big(\frac 12-a-b\big)\pi + 2b\arctan(\frac ac)+2a\arctan(\frac bc)+\arctan(\frac{c^2-a^2 b^2}{2abc})
\end{equation}
when $a=b=x, c=\sqrt{1-2a^2}, z=1,y=\sqrt{2}x$ and $x<\frac{\sqrt{2}}{2}$.

\bibliographystyle{plain}
\bibliography{ref.bib}
\end{document}



