\documentclass{article}
\usepackage{bm}
\usepackage{bbm}
\usepackage{amsmath}
\usepackage{amsthm}
\usepackage{algorithm,algorithmic}
\usepackage{url}
\usepackage{amssymb}
\usepackage{mathtools}
\usepackage{hyperref}
\newtheorem{theorem}{Theorem}
\newtheorem{example}{Example}
\title{A stochastic convex hull problem}
\begin{document}
\maketitle
\section{Two dimensions}
$(x_i, y_i) \sim p(x,y)$ are i.i.d. distributions
for $i=1,2,\dots, n, n+1$. We consider the event
$A_n =\{(x_{n+1}, y_{n+1}) \in ConvexHull\{(x_1, y_1),
\dots, (x_n, y_n)\} \}$. We give a lower bound on $A_n^c$
(complement) as follows:
\begin{theorem}\label{thm:lower_bound}
    \begin{equation}
        P(A_n^c) \geq \frac{2}{n+1}
    \end{equation}
\end{theorem}
\begin{proof}
We use the method of projection to prove Theorem
\ref{thm:lower_bound}.
Consider a direction $(\cos\theta, \sin \theta)$
characterized by $\theta$. The projection of $(x_i, y_i)$
is given by $x_i \cos\theta + y_i \sin \theta$.
If $\exists \theta$ such that
$x_{n+1}\cos\theta + y_{n+1} \sin \theta$
is larger than $\max_{i=1,\dots, n} \{x_{i}\cos\theta +
y_{i} \sin \theta\}$ or smaller than
$\min_{i=1,\dots, n} \{x_{i}\cos\theta +
y_{i} \sin \theta\}$, $A_n^c$ happens.
Let $z_i = x_i \cos \theta + y_i \sin \theta$,
$z_{\max}=\max\{z_1, \dots, z_n\}$ and
$z_{\min}=\min\{z_1, \dots, z_n\}$.
Then we have $P(A_n^c) \geq P(\exists \theta, z_{n+1} > z_{\max})
+ P(\exists \theta, z_{n+1} < z_{\min})
\geq P(\theta=\theta', z_{n+1} > z_{\max})
+ P(\theta=\theta', z_{n+1} < z_{\min}) $.

Now we compute $P(z_{n+1} > z_{\max})$
for a number $\theta'$.
The PDF of $z=x\cos \theta' + y \sin \theta'$
is given as $p(z)=\int
\frac{p(x, \frac{z-x \cos \theta'}
{\sin \theta'})}{|\sin \theta'|}dx$.
$P(z_{n+1} > z_{\max}) = \int P(z_{\max} < z)p(z)dz
= \int F(z)^n p(z)dz$ where $F(z)$ is the CDF of the
random variable $Z$. Therefore, the integral evaluates
to $\frac{1}{n+1}$. Similarly,
$P(\theta=\theta', z_{n+1} < z_{\min})=
\frac{1}{n+1}$.
\end{proof}

\section{Crofton's formula}
This part mainly consults \cite{crofton}.

Crofton's formula gives a way to compute the arc length
by integration. Suppose we have a two dimensional
parametrized curve $\gamma: [0, 1] \to \mathbb{R}^2$.
Then the arc length of $\gamma$ is
\begin{equation}\label{eq:gamma}
    len(\gamma) = \frac{1}{2}\iint n(p, \theta)
    dp d\theta
\end{equation}
where $p, \theta$ is the polar coordinate representation
of a straight line. $n(p, \theta)$ is the number of times the 
straight line intersects the curve. The integration
in \eqref{eq:gamma} is over $p\in [0, +\infty)$
and $\theta \in [0, 2\pi]$.
\begin{example}
    The perimeter of the circle $x^2+y^2=r^2$.
    Using the integration formula \eqref{eq:gamma},
    we obtain $\frac{1}{2}\int_{0}^{2\pi} d\theta \int_0^r 2 dp=2\pi r$.
\end{example}
\begin{example}
    The line segment $0\leq x \leq 1, y=0$.
    The integral is $\frac{1}{2}\int_0^1 2 dp
    \int_0^{\arccos p} d\theta = \int_0^1 \arccos p dp=1$.
\end{example}
\begin{thebibliography}{9}
    \bibitem{crofton} \url{https://math.osu.edu/sites/math.osu.edu/files/What%20is%202017%20Croftons%20Formula.pdf}
\end{thebibliography}
\end{document}



