\documentclass{article}
\usepackage{ctex}
\usepackage{amsfonts}
\usepackage{amsmath}
\usepackage{amsthm}
\usepackage{bm}
\DeclareMathOperator\E{\mathbb{E}}
\theoremstyle{definition}
\newtheorem{definition}{定义}
\newtheorem{example}{例}
\usepackage{mathtools}
\DeclarePairedDelimiter\abs{\lvert}{\rvert}
\begin{document}
\title{PAC笔记}
\author{zhaofeng-shu33}
\maketitle
\section{Basics}
\begin{definition}[PAC 可学习的概念]
称一个概念类$C$是 PAC 可学习的,如果存在一个算法 $A$ 使得$\forall \epsilon>0, \delta>0, \mathcal{X}$ 上的分布$D$,和目标概念$c \in C$, 当$ m \geq \textrm{poly}(1/\epsilon, 1/\delta, \textrm{size}(\bm{x}), \textrm{size}(c))$
\end{definition}
设 $ \mathcal{X} = \mathbb{R}$ 并考虑 区间模型 $ c(x) = I[a<x<b]$。我们将说明这个模型类是 PAC可学习的,样本复杂度为
$ m > {2 \over \epsilon} \ln\left({2 \over \delta} \right) $
\begin{example}
设$X \sim D$,对于实轴上$m$个样本,取包含所有正类样本点的最小区间作为$h_s: x \to I[a' < x < b']$,泛化误差为将正类样本点错分为负类样本点。
只需说明下式成立
\begin{equation}\label{eq:PAC}
\Pr_{S \sim D^m} [ R(h_s) > \epsilon ] < \delta
\end{equation}

若 $\Pr(a < X < b) \leq \epsilon $, 正类样本点出现是小概率事件,~\eqref{eq:PAC} 式左端为零,自然满足。
对于$\Pr(a < X < b) > \epsilon $,取 $ a^* < b^* $ 使得 $\Pr(a<x<a^*) = {\epsilon \over 2}, \Pr( b^*<x< b) = {\epsilon \over 2}$
事件$R(h_s) > \epsilon$ 包含在事件没有点落在$(a,a^*)$ 中或$(b,b^*)$中。因为如果$m$个样本点有落在$(a,a^*)$还有落在$(b,b^*)$中的,那么$a'\in (a,a^*), b'\in (b,b^*)$ 分错概率为$\Pr(a<X<a')+\Pr(b'<X<b)<\epsilon $。所以
\begin{align}
\Pr_{S \sim D^m} [ R(h_s) > \epsilon ]  & < 2 \prod_{i=1}^m \Pr(X_i < a \textrm{ or } X_i>a^*) \\
& = 2(1-{\epsilon \over 2})^m < 2 \exp({-\epsilon m \over 2 })
\end{align}
\end{example}

对于假设$h$取值空间$\mathcal{H}$ 是有限的情形,若要学习的概念$c \in \mathcal{H}$,可以说明
\begin{equation}
m \geq {1 \over \epsilon} ( \ln \abs{\mathcal{H}} + \ln{ 1 \over \delta})
\end{equation}
\section{PAC Bayes}
PAC Bayes 方法用于给损失函数定界。
\end{document}
