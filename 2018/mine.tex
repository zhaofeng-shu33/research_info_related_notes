\documentclass{article}
\usepackage{ctex}
\usepackage{amsfonts}
\usepackage{amsmath}
\usepackage{amsthm}
\usepackage{bm}
\DeclareMathOperator\E{\mathbb{E}}
\DeclareMathOperator\Var{\mathrm{Var}}
\newtheorem{lemma}{引理}
\theoremstyle{definition}
\newtheorem{definition}{定义}
\newtheorem{example}{例}
\usepackage{mathtools}
\def\P{\mathbb{P}}
\def\Q{\mathbb{Q}}
\DeclarePairedDelimiter\abs{\lvert}{\rvert}
\begin{document}
\title{用神经网络计算互信息}
\author{zhaofeng-shu33}
\maketitle
\section{背景}
K-L 散度的Donsker-Varadhan 表示:
\begin{equation}
D_{KL} (\P || \Q) = \sup_{T: \Omega \to R} \E_{\P}[T] - \log (\E_{\Q}[e^T])
\end{equation}
于是对于函数族 $ \mathcal{F}$ ,有K-L 散度的下界表示:
\begin{equation}
D_{KL} (\P || \Q) \geq \sup_{T \in \mathcal{F} } \E_{\P}[T] - \log (\E_{\Q}[e^T])
\end{equation}
\section{方法}
选取神经网络可表示的函数族,利用$ I(X; Z) = D_{KL}(\P_{XZ} || \P_X \otimes \P_Z) $,得到互信息的下界估计
\begin{equation}
I(X; Z) \geq I_{\Theta}(X; Z)
\end{equation}
其中
\begin{equation}
I_{\Theta}(X; Z) = \sup_{\theta \in \Theta} \E_{\P_{XZ}}[T_{\theta}] - \log ( \E_{\P_X \otimes \P_Z} [\exp(T_{\theta})])
\end{equation}
更复杂的神经网络给出更紧的下界估计,可以用SGD优化神经网络的权系数参数,求期望用样本均值代替。

\end{document}
